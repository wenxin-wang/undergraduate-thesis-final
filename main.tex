%%% -*- TeX-engine: xetex -*-
\documentclass{beamer}

% for themes, etc.
\input{preambles/packages}
\input{preambles/styles}
\input{preambles/fonts}

\title{
  基于翻译技术的流量调度研究 \\
  结题报告
}
\author{王文鑫}
\date{2016年6月8日}

\begin{document}

\begin{frame}
  \titlepage
\end{frame}

\section{研究课题}

\begin{frame}
  \frametitle{研究课题}

  \begin{block}{IPv4}
  在IPv4/IPv6过渡场景中,基于翻译等过渡技术,\\设计和实现灵活通用的流量调度机制,\\
  使得ISP可以自由定制多出口情景下的调度策略
  \end{block}
\end{frame}

\section{研究背景}
\subsection{IPv4/IPv6过渡}
\begin{frame}
  \frametitle{IPv4和IPv6}

  \begin{block}{IPv4}
    \begin{itemize}
    \item 地址数量有限:$2^{32} \doteq 43$亿地址,已于2011年2月3日枯竭\footnotemark[1]
    \end{itemize}
  \end{block}

  \begin{block}{IPv6:吸取了IPv4的经验和教训}
    \begin{itemize}
    \item 地址数量巨大:$2^{128} \doteq 10^{38}$个地址
    \item 简化数据格式、配置和路由表,提高效率:
      \begin{itemize}
      \item IP包头:去除了校验码等信息,提高处理效率
      \item 地址配置:利用设备标识自动配置全局可达地址,简化管理
      \item 全局路由表:采用高度聚类的前缀分配原则,提高路由匹配效率
      \end{itemize}
    \item 原生的多播支持、端到端透明、强制IPSec支持……
    \end{itemize}
  \end{block}
  \footnotetext[1]{https://www.nro.net/news/ipv4-free-pool-depleted}
\end{frame}

\begin{frame}
  \frametitle{过渡技术}

  \begin{block}{IPv6不兼容IPv4}
    \begin{itemize}
    \item IPv4到IPv6的切换是一个渐进的过程
      \begin{itemize}
      \item IPv6诞生20周年:扩展迅速,全球覆盖率未过半\footnotemark[1]
      \end{itemize}
    \item 在很长的一段时间内,IPv4和IPv6网络共存
    \end{itemize}
  \end{block}
  \begin{block}{过渡技术}
    \begin{itemize}
    \item 保证升级过程中对原始IPv4应用的支持
    \item 跨越不同IP协议访问网络资源
    \end{itemize}
  \end{block}
  \footnotetext[1]{https://www.google.com/intl/en/ipv6/statistics.html}
\end{frame}

\begin{frame}
  \frametitle{双栈、封装和翻译}

  \vspace{-1em}
  \begin{columns}[T] % align columns
    \begin{column}{.29\textwidth}
      \begin{block}{双栈}
        \begin{itemize}
        \item 不互通的两个协议栈
        \item IPv4地址短缺
        \end{itemize}
      \end{block}
    \end{column}
    \hfill
    \begin{column}{.70\textwidth}
    \end{column}
  \end{columns}
  \begin{columns}[T]
    \begin{column}{.29\textwidth}
      \begin{block}{封装}
        \begin{itemize}
        \item 一种IP协议数据包封装另一种
        \item 协议不互通
        \end{itemize}
      \end{block}
    \end{column}
    \hfill
    \begin{column}{.70\textwidth}
    \end{column}
  \end{columns}
  \begin{columns}[T]
    \begin{column}{.29\textwidth}
      \begin{block}{翻译}
        \begin{itemize}
        \item 对IP包头进行翻译协议
        \item 协议间互通
        \end{itemize}
      \end{block}
    \end{column}
    \hfill
    \begin{column}{.70\textwidth}
    \end{column}
  \end{columns}
\end{frame}

\begin{frame}
  \frametitle{课题研究的过渡场景}

  \begin{block}{使用IPv6主干网承载IPv4端到端通信}
    \begin{itemize}
    \item 用户端和资源端均存在大量不支持IPv6协议的应用
    \item ISP的IPv6网络越来越成熟
    \item 翻译技术:DIVI、NAT64
    \item 封装技术:DS-Lite
    \end{itemize}
  \end{block}
\end{frame}

\begin{frame}
  \frametitle{DIVI、NAT64、DS-Lite}

\end{frame}

\subsection{多出口流量调度}
\begin{frame}
  \frametitle{多出口流量调度}

  \begin{block}{多出口的作用}
    \begin{itemize}
    \item 优化带宽利用率、成本,主备切换……
    \item 搭建不同过渡系统,优势互补
    \item 按照一定的策略引导用户流量进入不同的上游网络
    \end{itemize}
  \end{block}

  \begin{block}{调度机制}
    \begin{itemize}
    \item 传统IPv4网络:BGP、策略路由
    \item 过渡场景:用户的IPv4流量先在用户接入口汇聚
    \item 将流量导向不同的过渡系统,从而选择上游出口
    \end{itemize}
  \end{block}
\end{frame}

\subsection{课题工作内容}
\begin{frame}
  \frametitle{工作内容}

  \begin{block}{工作}
    \begin{itemize}
    \item 调度机制设计和实现
      \begin{itemize}
      \item 灵活:基于五元组
      \item 通用:与具体过渡技术解耦
      \end{itemize}
    \item 修改现有过渡技术,适配多出口场景
    \item 测试和验证
    \end{itemize}
  \end{block}

  \begin{block}{意义}
    \begin{itemize}
    \item 使得过渡场景下的流量调度应用设计和开发成为可能
    \item ISP通过制定调度策略,优化用户体验和管理成本
    \end{itemize}
  \end{block}
\end{frame}

\section{设计和实现}
\begin{frame}
  \frametitle{总体结构}

  \begin{block}{设计的核心:用户接入口}
    \begin{itemize}
    \item 上游出口分配给各一级处理模块
    \item 用户网接入点设置各个二级处理模块
    \item 接入点按照五元组等信息分发流量到二级处理模块
    \item 二级处理模块交给各自的一级处理模块
    \end{itemize}
  \end{block}
\end{frame}

\begin{frame}
  \frametitle{设计原则}

  \begin{block}{正确性}
    \begin{itemize}
    \item 二级处理模块和调度机制的共存
    \item 用户IPv6接入与调度机制的IPv6接入共存
    \item 规则设计完备且尽量小巧
    \end{itemize}
  \end{block}

  \begin{block}{高效性}
    \begin{itemize}
    \item 用户流量在千兆或者万兆级别,间接代价被流量放大
    \item 牺牲代码和配置的可读性,换取性能
    \end{itemize}
  \end{block}

  \begin{block}{通用性}
    \begin{itemize}
    \item 不涉及任何二级处理模块的内部功能
    \item 对二级处理模块实现方式没有任何假设
    \item 不同二级处理模块平级对待
    \end{itemize}
  \end{block}
\end{frame}

\begin{frame}
  \frametitle{用户网接入点结构设计}
  \begin{block}{用户IPv4接入网口}
    \begin{itemize}
    \item DHCP服务:分发用户IPv4地址
    \item DNS服务:域名解析
      \begin{itemize}
      \item 不涉及地址翻译,多级查询依赖过渡技术
      \end{itemize}
    \end{itemize}
  \end{block}

  \begin{block}{用户IPv6接入网口}
    \begin{itemize}
    \item 不受限制:RA、DHCPv6、DNS
    \end{itemize}
  \end{block}
\end{frame}

\begin{frame}
  \frametitle{用户网接入点结构设计}
  \begin{block}{判断模块}
    \begin{itemize}
    \item 根据五元组信息,判断数据包发往某个二级处理模块
    \item 通过标记数据包实现(见后)
      \begin{itemize}
      \item 识别二级处理模块退回的数据包
      \end{itemize}
    \end{itemize}
  \end{block}

  \begin{block}{分发模块}
    \begin{itemize}
    \item 根据决策,将数据包发往不同的二级处理模块
      \begin{itemize}
      \item 如果调度策略和传统路由类似,可以和判断模块合并(见后) 
      \end{itemize}
    \end{itemize}
  \end{block}
\end{frame}

\begin{frame}
  \frametitle{用户网接入点结构设计}
  \begin{block}{各二级处理模块}
    \begin{itemize}
    \item 需要为调度机制提供可控的数据包传入口
    \item 二级模块间路由等规则的冲突解决
    \end{itemize}
  \end{block}

  \begin{block}{IPv6主干网接入网口}
    \begin{itemize}
    \item 设备接口有限,所有二级模块共享一个主干网接入网口
    \end{itemize}
  \end{block}
\end{frame}

\section{完成工作}
\begin{frame}
  \frametitle{完成工作}

  \begin{block}{}
    \begin{itemize}
    \item 使用libvirt和netns搭建环境,测试拓扑设计和代码
    \item OVS:设计流表
    \item 翻译器:netfilter $\rightarrow$ iptables,可以被netns隔离
    \item 控制器:静态策略、轮询切换
    \end{itemize}
  \end{block}
\end{frame}

\section{流量调度机制设计}

\begin{frame}
  \frametitle{实验平台}

  \begin{center}
    \includegraphics[width=\textwidth]{figs/test-env.jpeg}
  \end{center}
\end{frame}

\begin{frame}
  \frametitle{实验平台}

  \begin{center}
    \includegraphics[width=\textwidth]{figs/test-env.jpeg}
  \end{center}
\end{frame}

\begin{frame}
  \frametitle{控制器}

  \begin{center}
    \includegraphics[width=\textwidth]{figs/static.jpeg}
  \end{center}

  \begin{center}
    \includegraphics[width=\textwidth]{figs/round-robin.jpeg}  
  \end{center}
  
\end{frame}

\section{后续工作}
\begin{frame}
  \frametitle{后续工作}

  \begin{block}{}
    \begin{itemize}
    \item 在多个二级翻译器间控制调度
    \item 添加NAT64隧道
    \end{itemize}
  \end{block}
\end{frame}

\section{进度安排}

\begin{frame}
  \frametitle{进度安排}
  \begin{block}{}
    \begin{itemize}
    \item 4月15日-5月15日:实现多个翻译器间的调度
    \item 5月16日-答辩:封装接口,完善论文
    \end{itemize}
  \end{block}
\end{frame}

\section{参考文献}
\begin{frame}
  \frametitle{参考文献}
  \begin{itemize}
  \item \href{https://tools.ietf.org/html/rfc7599}{RFC7599}
  \item \href{https://tools.ietf.org/html/rfc7597}{RFC7597}
  \item \href{https://tools.ietf.org/html/rfc6219}{RFC6219}
  \item \href{https://tools.ietf.org/html/rfc6052}{RFC6052}
  \item \href{https://tools.ietf.org/html/rfc7598}{RFC7598}
  \item
    \href{http://openvswitch.org/}{Open vSwitch}
  \item
    \href{http://osrg.github.io/ryu/}{Ryu component-based software defined networking framework }
  \end{itemize}
\end{frame}

\section{Q\&A}

\begin{frame}
  \frametitle{Q\&A}
  \begin{center}
    {\LARGE 谢谢!}
    \vspace{3em}
    {\LARGE 请老师们提问和指导!}
  \end{center}
\end{frame}
\end{document}